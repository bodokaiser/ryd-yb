\usepackage{amsmath}
\usepackage{booktabs}
\usepackage{xcolor}
\usepackage[english]{babel}
%\usepackage{unicode-math}
\usepackage{mathtools}
%\usepackage{derivative}
%\usepackage{makecell}
%\usepackage{multirow}
\usepackage{siunitx}
%\usepackage{pgfplots}
\usepackage{circuitikz}
%\usepackage{appendixnumberbeamer}

\usetheme{metropolis}

%\setmainfont{Stix Two Text}
%\setmathfont{Stix Two Math}

%\pgfplotsset{compat=1.17}
\renewcommand{\familydefault}{\sfdefault}
\usetikzlibrary{arrows.meta}
\usetikzlibrary{fit}
\usetikzlibrary{positioning}
%\usetikzlibrary{shapes.geometric}
%\usetikzlibrary{shapes.misc}
%\usepgfplotslibrary{groupplots}

\DeclarePairedDelimiter{\ceil}{\lceil}{\rceil}
\DeclarePairedDelimiter{\floor}{\lfloor}{\rfloor}
\DeclarePairedDelimiter{\abs}{\lvert}{\rvert}
\DeclarePairedDelimiter{\norm}{\lVert}{\rVert}
\DeclarePairedDelimiter{\bra}{\langle}{\rvert}
\DeclarePairedDelimiter{\ket}{\lvert}{\rangle}
\DeclarePairedDelimiter{\expval}{\langle}{\rangle}
\DeclarePairedDelimiter{\norder}{\mathcolon}{\mathcolon}
\DeclarePairedDelimiter{\anorder}{\typecolon}{\typecolon}
	
\newcommand{\laplace}{\mbfnabla^2}
\newcommand{\trans}{{\scriptscriptstyle\mathsf{T}}}

\newcommand{\conv}{\ast}
\newcommand{\vdot}{\cdot}
\newcommand{\vcross}{\vectimes}
\newcommand{\vb}[1]{\symbfup{#1}}
\newcommand{\vu}[1]{\hat{\vb{#1}}}
\newcommand*\dd[2][\relax]{\mathop{\ifx\relax#1\odif{#2}\else \odif[order={#1}]{#2}\fi}}

\newcommand{\vacket}{\ket*{0}}
\newcommand{\vacbra}{\bra*{0}}

\DeclareMathOperator{\trace}{Tr}
\DeclareMathOperator{\sinc}{sinc}

\AtBeginDocument{
	\let\Re\relax
	\let\Im\relax
	\DeclareMathOperator{\Re}{Re}
	\DeclareMathOperator{\Im}{Im}

	\renewcommand{\div}{\mathop{\mbfnabla\vdot}}
	\newcommand{\curl}{\mathop{\mbfnabla\vectimes}}
}

\DeclarePairedDelimiterX{\comm}[2]{[}{]}{#1,#2}

\DeclarePairedDelimiterX{\braket}[2]{\langle}{\rangle}{#1\delimsize\vert#2}
\DeclarePairedDelimiterX{\ketbra}[1]{\lvert}{\rvert}{#1\rangle\delimsize\langle#1}
